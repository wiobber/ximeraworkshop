\documentclass{ximera}
%\input{../../preamble.tex}
%\addPrintStyle{../..}
\begin{document}
     
    \author{Zomercursus KU Leuven}
 
    \xmtitle{Oefeningen  breuken}

{\color{red}Het zou fijn zijn om bij deze oefeningen feedback of een oplossing te voorzien. Het verschil tussen beide is dat feedback onmiddellijk zichtbaar wordt wanner de student de oefening heeft proberen op te lossen. Om de oplossing te zien moet de student nog klikken op "Toon uitwerking". Beide voeg je toe met een omgeving \verb|\begin{feedback/oplossing} ... \end{feedback/oplossing}|. Probeer ook eens een hint toe te voegen op dezelfde manier.}

{\color{red} Probeer ook eens een multiple choice oefening te maken. Kies je voor een lijst of een drop down menu? Voor inspiratie kan je een kijkje nemen op \url{https://set.kuleuven.be/ximera-wis/demo/auteur/auteurs/ximeraDemo}. Als je in de zoekbalk achter deze link .tex toevoegt, kan je de texcode zien en vergelijken met de live versie.}

 \xmsection{Niveau 1}
  
\begin{exercise}
    Schrijf zo eenvoudig mogelijk.
    \begin{question}$20\cdot(\frac{5}{4}-\frac{4}{5})=\answer[format=integer,onlineshowanswerbutton]{9}$\end{question}
    \begin{question}$-\frac{6}{27}+\frac{27}{1} + \frac{16+14}{9} -\frac{3}{14+13} - 3\cdot9=\answer[format=integer,onlineshowanswerbutton]{3}$\end{question}
    \begin{question}$-\frac{(1-a)-2}{a+1}=\answer[format=integer,onlineshowanswerbutton]{1}$\hfill($a\neq-1$)\end{question}
\end{exercise}
 
 \xmsection{Niveau 2}
  
\begin{exercise}
Schrijf als een zo eenvoudig mogelijke breuk. Stel $a,b,c \in \Rnul$.
\begin{question}$\frac{a-b}{c}-\frac{a-2b}{2c}=\answer[onlineshowanswerbutton]{\frac{a}{2c}}$\end{question}
\begin{question}$\frac{\frac{a-b}{b}}{1-\frac{a}{b}}=\answer[onlineshowanswerbutton]{-1}$\end{question}
\begin{question}$\frac{1-\frac{a+b}{b}}{\frac{a^{2}}{b}}=\answer[onlineshowanswerbutton]{-\frac{1}{a}}$\end{question}
\begin{question}$a+\cfrac{a}{1+a}=\answer[onlineshowanswerbutton]{\frac{2a+a^2}{1+a}}$\end{question}   
\begin{question}$1+\cfrac{a}{1+a}=\answer[onlineshowanswerbutton]{\frac{1+2a}{1+a}}$\end{question} 
\begin{question}$1+\cfrac{1}{1+a}=\answer[onlineshowanswerbutton]{\frac{2+a}{1+a}}$\end{question}  
\begin{question}$a+\cfrac{1}{1+a}=\answer[onlineshowanswerbutton]{\frac{1+a+a^2}{1+a}}$\end{question}  
\begin{question}$\cfrac{1}{1+\cfrac{1}{1+a}}=\answer[onlineshowanswerbutton]{\frac{1+a}{2+a}}$\end{question}
\begin{question}$\cfrac{1}{1+\cfrac{1}{1+\cfrac{1}{1+a}}}=\answer[onlineshowanswerbutton]{\frac{2+a}{3+2a}}$\end{question}
\end{exercise}
 
 
\end{document}